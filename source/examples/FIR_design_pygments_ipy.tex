\documentclass[a4paper,11pt,final]{article}
\usepackage{fancyvrb, color, graphicx, hyperref, amsmath, url}
\usepackage{palatino}
\usepackage{pygments}
\usepackage[a4paper,text={16.5cm,25.2cm},centering]{geometry}
        
\hypersetup  
{   pdfauthor = {Matti Pastell},
  pdftitle={FIR filter design with Python and SciPy},
  colorlinks=TRUE,
  linkcolor=black,
  citecolor=blue,
  urlcolor=blue
}

\setlength{\parindent}{0pt}
\setlength{\parskip}{1.2ex}


        
\title{FIR filter design with Python and SciPy}
\author{Matti Pastell \\ \url{http://mpastell.com}}
\date{15th April 2013}

\begin{document}
\maketitle


\section{Introduction}

This an example of a document that can be published using
\href{http://mpastell.com/pweave}{Pweave}. Text is written
using \LaTeX{} and code between \texttt{<<>>} and \texttt{@} is executed
and results are included in the resulting document.

You can define various options for code chunks to control code
execution and formatting (see
\href{http://mpastell.com/pweave/usage.html\#code-chunk-options}{Pweave
docs}).

\section{FIR Filter Design}

We'll implement lowpass, highpass and ' bandpass FIR filters. If you
want to read more about DSP I highly recommend
\href{http://www.dspguide.com/}{The Scientist and Engineer's Guide to
Digital Signal Processing} which is freely available online.

\subsection{Functions for frequency, phase, impulse and step response}

Let's first define functions to plot filter properties.




\begin{Verbatim}[commandchars=\\\{\},frame=single,fontsize=\small, xleftmargin=0.5em]
\PY{k+kn}{from} \PY{n+nn}{pylab} \PY{k+kn}{import} \PY{o}{*}
\PY{k+kn}{import} \PY{n+nn}{scipy.signal} \PY{k+kn}{as} \PY{n+nn}{signal}
    
\PY{c+c1}{\PYZsh{}Plot frequency and phase response}
\PY{k}{def} \PY{n+nf}{mfreqz}\PY{p}{(}\PY{n}{b}\PY{p}{,}\PY{n}{a}\PY{o}{=}\PY{l+m+mi}{1}\PY{p}{)}\PY{p}{:}
    \PY{n}{w}\PY{p}{,}\PY{n}{h} \PY{o}{=} \PY{n}{signal}\PY{o}{.}\PY{n}{freqz}\PY{p}{(}\PY{n}{b}\PY{p}{,}\PY{n}{a}\PY{p}{)}
    \PY{n}{h\PYZus{}dB} \PY{o}{=} \PY{l+m+mi}{20} \PY{o}{*} \PY{n}{log10} \PY{p}{(}\PY{n+nb}{abs}\PY{p}{(}\PY{n}{h}\PY{p}{)}\PY{p}{)}
    \PY{n}{subplot}\PY{p}{(}\PY{l+m+mi}{211}\PY{p}{)}
    \PY{n}{plot}\PY{p}{(}\PY{n}{w}\PY{o}{/}\PY{n+nb}{max}\PY{p}{(}\PY{n}{w}\PY{p}{)}\PY{p}{,}\PY{n}{h\PYZus{}dB}\PY{p}{)}
    \PY{n}{ylim}\PY{p}{(}\PY{o}{\PYZhy{}}\PY{l+m+mi}{150}\PY{p}{,} \PY{l+m+mi}{5}\PY{p}{)}
    \PY{n}{ylabel}\PY{p}{(}\PY{l+s+s1}{\PYZsq{}}\PY{l+s+s1}{Magnitude (db)}\PY{l+s+s1}{\PYZsq{}}\PY{p}{)}
    \PY{n}{xlabel}\PY{p}{(}\PY{l+s+sa}{r}\PY{l+s+s1}{\PYZsq{}}\PY{l+s+s1}{Normalized Frequency (x\PYZdl{}}\PY{l+s+s1}{\PYZbs{}}\PY{l+s+s1}{pi\PYZdl{}rad/sample)}\PY{l+s+s1}{\PYZsq{}}\PY{p}{)}
    \PY{n}{title}\PY{p}{(}\PY{l+s+sa}{r}\PY{l+s+s1}{\PYZsq{}}\PY{l+s+s1}{Frequency response}\PY{l+s+s1}{\PYZsq{}}\PY{p}{)}
    \PY{n}{subplot}\PY{p}{(}\PY{l+m+mi}{212}\PY{p}{)}
    \PY{n}{h\PYZus{}Phase} \PY{o}{=} \PY{n}{unwrap}\PY{p}{(}\PY{n}{arctan2}\PY{p}{(}\PY{n}{imag}\PY{p}{(}\PY{n}{h}\PY{p}{)}\PY{p}{,}\PY{n}{real}\PY{p}{(}\PY{n}{h}\PY{p}{)}\PY{p}{)}\PY{p}{)}
    \PY{n}{plot}\PY{p}{(}\PY{n}{w}\PY{o}{/}\PY{n+nb}{max}\PY{p}{(}\PY{n}{w}\PY{p}{)}\PY{p}{,}\PY{n}{h\PYZus{}Phase}\PY{p}{)}
    \PY{n}{ylabel}\PY{p}{(}\PY{l+s+s1}{\PYZsq{}}\PY{l+s+s1}{Phase (radians)}\PY{l+s+s1}{\PYZsq{}}\PY{p}{)}
    \PY{n}{xlabel}\PY{p}{(}\PY{l+s+sa}{r}\PY{l+s+s1}{\PYZsq{}}\PY{l+s+s1}{Normalized Frequency (x\PYZdl{}}\PY{l+s+s1}{\PYZbs{}}\PY{l+s+s1}{pi\PYZdl{}rad/sample)}\PY{l+s+s1}{\PYZsq{}}\PY{p}{)}
    \PY{n}{title}\PY{p}{(}\PY{l+s+sa}{r}\PY{l+s+s1}{\PYZsq{}}\PY{l+s+s1}{Phase response}\PY{l+s+s1}{\PYZsq{}}\PY{p}{)}
    \PY{n}{subplots\PYZus{}adjust}\PY{p}{(}\PY{n}{hspace}\PY{o}{=}\PY{l+m+mf}{0.5}\PY{p}{)}

\PY{c+c1}{\PYZsh{}Plot step and impulse response}
\PY{k}{def} \PY{n+nf}{impz}\PY{p}{(}\PY{n}{b}\PY{p}{,}\PY{n}{a}\PY{o}{=}\PY{l+m+mi}{1}\PY{p}{)}\PY{p}{:}
    \PY{n}{l} \PY{o}{=} \PY{n+nb}{len}\PY{p}{(}\PY{n}{b}\PY{p}{)}
    \PY{n}{impulse} \PY{o}{=} \PY{n}{repeat}\PY{p}{(}\PY{l+m+mf}{0.}\PY{p}{,}\PY{n}{l}\PY{p}{)}\PY{p}{;} \PY{n}{impulse}\PY{p}{[}\PY{l+m+mi}{0}\PY{p}{]} \PY{o}{=}\PY{l+m+mf}{1.}
    \PY{n}{x} \PY{o}{=} \PY{n}{arange}\PY{p}{(}\PY{l+m+mi}{0}\PY{p}{,}\PY{n}{l}\PY{p}{)}
    \PY{n}{response} \PY{o}{=} \PY{n}{signal}\PY{o}{.}\PY{n}{lfilter}\PY{p}{(}\PY{n}{b}\PY{p}{,}\PY{n}{a}\PY{p}{,}\PY{n}{impulse}\PY{p}{)}
    \PY{n}{subplot}\PY{p}{(}\PY{l+m+mi}{211}\PY{p}{)}
    \PY{n}{stem}\PY{p}{(}\PY{n}{x}\PY{p}{,} \PY{n}{response}\PY{p}{)}
    \PY{n}{ylabel}\PY{p}{(}\PY{l+s+s1}{\PYZsq{}}\PY{l+s+s1}{Amplitude}\PY{l+s+s1}{\PYZsq{}}\PY{p}{)}
    \PY{n}{xlabel}\PY{p}{(}\PY{l+s+sa}{r}\PY{l+s+s1}{\PYZsq{}}\PY{l+s+s1}{n (samples)}\PY{l+s+s1}{\PYZsq{}}\PY{p}{)}
    \PY{n}{title}\PY{p}{(}\PY{l+s+sa}{r}\PY{l+s+s1}{\PYZsq{}}\PY{l+s+s1}{Impulse response}\PY{l+s+s1}{\PYZsq{}}\PY{p}{)}
    \PY{n}{subplot}\PY{p}{(}\PY{l+m+mi}{212}\PY{p}{)}
    \PY{n}{step} \PY{o}{=} \PY{n}{cumsum}\PY{p}{(}\PY{n}{response}\PY{p}{)}
    \PY{n}{stem}\PY{p}{(}\PY{n}{x}\PY{p}{,} \PY{n}{step}\PY{p}{)}
    \PY{n}{ylabel}\PY{p}{(}\PY{l+s+s1}{\PYZsq{}}\PY{l+s+s1}{Amplitude}\PY{l+s+s1}{\PYZsq{}}\PY{p}{)}
    \PY{n}{xlabel}\PY{p}{(}\PY{l+s+sa}{r}\PY{l+s+s1}{\PYZsq{}}\PY{l+s+s1}{n (samples)}\PY{l+s+s1}{\PYZsq{}}\PY{p}{)}
    \PY{n}{title}\PY{p}{(}\PY{l+s+sa}{r}\PY{l+s+s1}{\PYZsq{}}\PY{l+s+s1}{Step response}\PY{l+s+s1}{\PYZsq{}}\PY{p}{)}
    \PY{n}{subplots\PYZus{}adjust}\PY{p}{(}\PY{n}{hspace}\PY{o}{=}\PY{l+m+mf}{0.5}\PY{p}{)}
\end{Verbatim}


\subsection{Lowpass FIR filter}

Designing a lowpass FIR filter is very simple to do with SciPy, all you
need to do is to define the window length, cut off frequency and the
window.

The Hamming window is defined as:
$w(n) = \alpha - \beta\cos\frac{2\pi n}{N-1}$, where $\alpha=0.54$ and
$\beta=0.46$

The next code chunk is executed in term mode, see the source document
for syntax. Notice also that Pweave can now catch multiple
figures/code chunk.



\begin{Verbatim}[commandchars=\\\{\},frame=single,fontsize=\small, xleftmargin=0.5em]
\PY{n}{n} \PY{o}{=} \PY{l+m+mi}{61}
\PY{n}{a} \PY{o}{=} \PY{n}{signal}\PY{o}{.}\PY{n}{firwin}\PY{p}{(}\PY{n}{n}\PY{p}{,} \PY{n}{cutoff} \PY{o}{=} \PY{l+m+mf}{0.3}\PY{p}{,} \PY{n}{window} \PY{o}{=} \PY{l+s+s2}{\PYZdq{}}\PY{l+s+s2}{hamming}\PY{l+s+s2}{\PYZdq{}}\PY{p}{)}
\PY{c+c1}{\PYZsh{}Frequency and phase response}
\PY{n}{mfreqz}\PY{p}{(}\PY{n}{a}\PY{p}{)}
\end{Verbatim}
\includegraphics[width= \linewidth]{figures/FIR_design_pygments_figure2_1.pdf}


\begin{Verbatim}[commandchars=\\\{\},frame=single,fontsize=\small, xleftmargin=0.5em]
\PY{n}{show}\PY{p}{(}\PY{p}{)}
\PY{c+c1}{\PYZsh{}Impulse and step response}
\PY{n}{figure}\PY{p}{(}\PY{l+m+mi}{2}\PY{p}{)}
\end{Verbatim}

\begin{Verbatim}[commandchars=\\\{\},frame=leftline,fontsize=\small, xleftmargin=0.5em]
<matplotlib.figure.Figure at 0x7f643acb6d30>
\end{Verbatim}

\begin{Verbatim}[commandchars=\\\{\},frame=leftline,fontsize=\small, xleftmargin=0.5em]
<matplotlib.figure.Figure at 0x7f643acb6d30>
\end{Verbatim}


\begin{Verbatim}[commandchars=\\\{\},frame=single,fontsize=\small, xleftmargin=0.5em]
\PY{n}{impz}\PY{p}{(}\PY{n}{a}\PY{p}{)}
\end{Verbatim}
\includegraphics[width= \linewidth]{figures/FIR_design_pygments_figure2_1.pdf}


\begin{Verbatim}[commandchars=\\\{\},frame=single,fontsize=\small, xleftmargin=0.5em]
\PY{n}{show}\PY{p}{(}\PY{p}{)}
\end{Verbatim}


\subsection{Highpass FIR Filter}

Let's define a highpass FIR filter:



\begin{Verbatim}[commandchars=\\\{\},frame=single,fontsize=\small, xleftmargin=0.5em]
\PY{n}{n} \PY{o}{=} \PY{l+m+mi}{101}
\PY{n}{a} \PY{o}{=} \PY{n}{signal}\PY{o}{.}\PY{n}{firwin}\PY{p}{(}\PY{n}{n}\PY{p}{,} \PY{n}{cutoff} \PY{o}{=} \PY{l+m+mf}{0.3}\PY{p}{,} \PY{n}{window} \PY{o}{=} \PY{l+s+s2}{\PYZdq{}}\PY{l+s+s2}{hanning}\PY{l+s+s2}{\PYZdq{}}\PY{p}{,} \PY{n}{pass\PYZus{}zero}\PY{o}{=}\PY{n+nb+bp}{False}\PY{p}{)}
\PY{n}{mfreqz}\PY{p}{(}\PY{n}{a}\PY{p}{)}
\PY{n}{show}\PY{p}{(}\PY{p}{)}
\end{Verbatim}
\includegraphics[width= \linewidth]{figures/FIR_design_pygments_figure3_1.pdf}


\subsection{Bandpass FIR filter}

Notice that the plot has a caption defined in code chunk options.




\begin{Verbatim}[commandchars=\\\{\},frame=single,fontsize=\small, xleftmargin=0.5em]
\PY{n}{n} \PY{o}{=} \PY{l+m+mi}{1001}
\PY{n}{a} \PY{o}{=} \PY{n}{signal}\PY{o}{.}\PY{n}{firwin}\PY{p}{(}\PY{n}{n}\PY{p}{,} \PY{n}{cutoff} \PY{o}{=} \PY{p}{[}\PY{l+m+mf}{0.2}\PY{p}{,} \PY{l+m+mf}{0.5}\PY{p}{]}\PY{p}{,} \PY{n}{window} \PY{o}{=} \PY{l+s+s1}{\PYZsq{}}\PY{l+s+s1}{blackmanharris}\PY{l+s+s1}{\PYZsq{}}\PY{p}{,} \PY{n}{pass\PYZus{}zero} \PY{o}{=} \PY{n+nb+bp}{False}\PY{p}{)}
\PY{n}{mfreqz}\PY{p}{(}\PY{n}{a}\PY{p}{)}
\PY{n}{show}\PY{p}{(}\PY{p}{)}
\end{Verbatim}
\begin{figure}[htpb]
\center
\includegraphics[width= \linewidth]{figures/FIR_design_pygments_figure4_1.pdf}
\caption{Bandpass FIR filter.}
\label{fig:None}
\end{figure}


\end{document}
